\documentclass[]{article}
\usepackage{lmodern}
\usepackage{amssymb,amsmath}
\usepackage{ifxetex,ifluatex}
\usepackage{fixltx2e} % provides \textsubscript
\ifnum 0\ifxetex 1\fi\ifluatex 1\fi=0 % if pdftex
  \usepackage[T1]{fontenc}
  \usepackage[utf8]{inputenc}
\else % if luatex or xelatex
  \ifxetex
    \usepackage{mathspec}
  \else
    \usepackage{fontspec}
  \fi
  \defaultfontfeatures{Ligatures=TeX,Scale=MatchLowercase}
\fi
% use upquote if available, for straight quotes in verbatim environments
\IfFileExists{upquote.sty}{\usepackage{upquote}}{}
% use microtype if available
\IfFileExists{microtype.sty}{%
\usepackage{microtype}
\UseMicrotypeSet[protrusion]{basicmath} % disable protrusion for tt fonts
}{}
\usepackage[margin=1in]{geometry}
\usepackage{hyperref}
\hypersetup{unicode=true,
            pdftitle={Social Equity, Access and the New York School System},
            pdfauthor={Kunal Kerai},
            pdfborder={0 0 0},
            breaklinks=true}
\urlstyle{same}  % don't use monospace font for urls
\usepackage{color}
\usepackage{fancyvrb}
\newcommand{\VerbBar}{|}
\newcommand{\VERB}{\Verb[commandchars=\\\{\}]}
\DefineVerbatimEnvironment{Highlighting}{Verbatim}{commandchars=\\\{\}}
% Add ',fontsize=\small' for more characters per line
\usepackage{framed}
\definecolor{shadecolor}{RGB}{248,248,248}
\newenvironment{Shaded}{\begin{snugshade}}{\end{snugshade}}
\newcommand{\KeywordTok}[1]{\textcolor[rgb]{0.13,0.29,0.53}{\textbf{#1}}}
\newcommand{\DataTypeTok}[1]{\textcolor[rgb]{0.13,0.29,0.53}{#1}}
\newcommand{\DecValTok}[1]{\textcolor[rgb]{0.00,0.00,0.81}{#1}}
\newcommand{\BaseNTok}[1]{\textcolor[rgb]{0.00,0.00,0.81}{#1}}
\newcommand{\FloatTok}[1]{\textcolor[rgb]{0.00,0.00,0.81}{#1}}
\newcommand{\ConstantTok}[1]{\textcolor[rgb]{0.00,0.00,0.00}{#1}}
\newcommand{\CharTok}[1]{\textcolor[rgb]{0.31,0.60,0.02}{#1}}
\newcommand{\SpecialCharTok}[1]{\textcolor[rgb]{0.00,0.00,0.00}{#1}}
\newcommand{\StringTok}[1]{\textcolor[rgb]{0.31,0.60,0.02}{#1}}
\newcommand{\VerbatimStringTok}[1]{\textcolor[rgb]{0.31,0.60,0.02}{#1}}
\newcommand{\SpecialStringTok}[1]{\textcolor[rgb]{0.31,0.60,0.02}{#1}}
\newcommand{\ImportTok}[1]{#1}
\newcommand{\CommentTok}[1]{\textcolor[rgb]{0.56,0.35,0.01}{\textit{#1}}}
\newcommand{\DocumentationTok}[1]{\textcolor[rgb]{0.56,0.35,0.01}{\textbf{\textit{#1}}}}
\newcommand{\AnnotationTok}[1]{\textcolor[rgb]{0.56,0.35,0.01}{\textbf{\textit{#1}}}}
\newcommand{\CommentVarTok}[1]{\textcolor[rgb]{0.56,0.35,0.01}{\textbf{\textit{#1}}}}
\newcommand{\OtherTok}[1]{\textcolor[rgb]{0.56,0.35,0.01}{#1}}
\newcommand{\FunctionTok}[1]{\textcolor[rgb]{0.00,0.00,0.00}{#1}}
\newcommand{\VariableTok}[1]{\textcolor[rgb]{0.00,0.00,0.00}{#1}}
\newcommand{\ControlFlowTok}[1]{\textcolor[rgb]{0.13,0.29,0.53}{\textbf{#1}}}
\newcommand{\OperatorTok}[1]{\textcolor[rgb]{0.81,0.36,0.00}{\textbf{#1}}}
\newcommand{\BuiltInTok}[1]{#1}
\newcommand{\ExtensionTok}[1]{#1}
\newcommand{\PreprocessorTok}[1]{\textcolor[rgb]{0.56,0.35,0.01}{\textit{#1}}}
\newcommand{\AttributeTok}[1]{\textcolor[rgb]{0.77,0.63,0.00}{#1}}
\newcommand{\RegionMarkerTok}[1]{#1}
\newcommand{\InformationTok}[1]{\textcolor[rgb]{0.56,0.35,0.01}{\textbf{\textit{#1}}}}
\newcommand{\WarningTok}[1]{\textcolor[rgb]{0.56,0.35,0.01}{\textbf{\textit{#1}}}}
\newcommand{\AlertTok}[1]{\textcolor[rgb]{0.94,0.16,0.16}{#1}}
\newcommand{\ErrorTok}[1]{\textcolor[rgb]{0.64,0.00,0.00}{\textbf{#1}}}
\newcommand{\NormalTok}[1]{#1}
\usepackage{graphicx,grffile}
\makeatletter
\def\maxwidth{\ifdim\Gin@nat@width>\linewidth\linewidth\else\Gin@nat@width\fi}
\def\maxheight{\ifdim\Gin@nat@height>\textheight\textheight\else\Gin@nat@height\fi}
\makeatother
% Scale images if necessary, so that they will not overflow the page
% margins by default, and it is still possible to overwrite the defaults
% using explicit options in \includegraphics[width, height, ...]{}
\setkeys{Gin}{width=\maxwidth,height=\maxheight,keepaspectratio}
\IfFileExists{parskip.sty}{%
\usepackage{parskip}
}{% else
\setlength{\parindent}{0pt}
\setlength{\parskip}{6pt plus 2pt minus 1pt}
}
\setlength{\emergencystretch}{3em}  % prevent overfull lines
\providecommand{\tightlist}{%
  \setlength{\itemsep}{0pt}\setlength{\parskip}{0pt}}
\setcounter{secnumdepth}{0}
% Redefines (sub)paragraphs to behave more like sections
\ifx\paragraph\undefined\else
\let\oldparagraph\paragraph
\renewcommand{\paragraph}[1]{\oldparagraph{#1}\mbox{}}
\fi
\ifx\subparagraph\undefined\else
\let\oldsubparagraph\subparagraph
\renewcommand{\subparagraph}[1]{\oldsubparagraph{#1}\mbox{}}
\fi

%%% Use protect on footnotes to avoid problems with footnotes in titles
\let\rmarkdownfootnote\footnote%
\def\footnote{\protect\rmarkdownfootnote}

%%% Change title format to be more compact
\usepackage{titling}

% Create subtitle command for use in maketitle
\newcommand{\subtitle}[1]{
  \posttitle{
    \begin{center}\large#1\end{center}
    }
}

\setlength{\droptitle}{-2em}

  \title{Social Equity, Access and the New York School System}
    \pretitle{\vspace{\droptitle}\centering\huge}
  \posttitle{\par}
    \author{Kunal Kerai}
    \preauthor{\centering\large\emph}
  \postauthor{\par}
      \predate{\centering\large\emph}
  \postdate{\par}
    \date{August 29, 2018}


\begin{document}
\maketitle

\subsubsection{Loading Libraries and
Data}\label{loading-libraries-and-data}

Let's load some libraries we will use.

\begin{Shaded}
\begin{Highlighting}[]
\KeywordTok{library}\NormalTok{(tidyverse)}
\end{Highlighting}
\end{Shaded}

\begin{verbatim}
## -- Attaching packages ------------------------------------------------------------------------------------------------------ tidyverse 1.2.1 --
\end{verbatim}

\begin{verbatim}
## v ggplot2 3.0.0     v purrr   0.2.5
## v tibble  1.4.2     v dplyr   0.7.6
## v tidyr   0.8.1     v stringr 1.3.1
## v readr   1.1.1     v forcats 0.3.0
\end{verbatim}

\begin{verbatim}
## -- Conflicts --------------------------------------------------------------------------------------------------------- tidyverse_conflicts() --
## x dplyr::filter() masks stats::filter()
## x dplyr::lag()    masks stats::lag()
\end{verbatim}

\begin{Shaded}
\begin{Highlighting}[]
\KeywordTok{library}\NormalTok{(ggplot2)}
\end{Highlighting}
\end{Shaded}

Let's load our data from the table:

\begin{Shaded}
\begin{Highlighting}[]
\NormalTok{reg <-}\StringTok{ }\KeywordTok{read_csv}\NormalTok{(}\StringTok{"D5 SHSAT Registrations and Testers.csv"}\NormalTok{)}
\end{Highlighting}
\end{Shaded}

\begin{verbatim}
## Parsed with column specification:
## cols(
##   DBN = col_character(),
##   `School name` = col_character(),
##   `Year of SHST` = col_integer(),
##   `Grade level` = col_integer(),
##   `Enrollment on 10/31` = col_integer(),
##   `Number of students who registered for the SHSAT` = col_integer(),
##   `Number of students who took the SHSAT` = col_integer()
## )
\end{verbatim}

\begin{Shaded}
\begin{Highlighting}[]
\NormalTok{school <-}\StringTok{ }\KeywordTok{read_csv}\NormalTok{(}\StringTok{"2016 School Explorer.csv"}\NormalTok{)}
\end{Highlighting}
\end{Shaded}

\begin{verbatim}
## Parsed with column specification:
## cols(
##   .default = col_integer(),
##   `Adjusted Grade` = col_character(),
##   `New?` = col_character(),
##   `Other Location Code in LCGMS` = col_character(),
##   `School Name` = col_character(),
##   `SED Code` = col_double(),
##   `Location Code` = col_character(),
##   Latitude = col_double(),
##   Longitude = col_double(),
##   `Address (Full)` = col_character(),
##   City = col_character(),
##   Grades = col_character(),
##   `Grade Low` = col_character(),
##   `Grade High` = col_character(),
##   `Community School?` = col_character(),
##   `Economic Need Index` = col_character(),
##   `School Income Estimate` = col_character(),
##   `Percent ELL` = col_character(),
##   `Percent Asian` = col_character(),
##   `Percent Black` = col_character(),
##   `Percent Hispanic` = col_character()
##   # ... with 19 more columns
## )
\end{verbatim}

\begin{verbatim}
## See spec(...) for full column specifications.
\end{verbatim}

Before we proceed, let's take a look at the initial data.

\begin{verbatim}
## # A tibble: 6 x 7
##   DBN   `School name` `Year of SHST` `Grade level` `Enrollment on ~
##   <chr> <chr>                  <int>         <int>            <int>
## 1 05M0~ P.S. 046 Art~           2013             8               91
## 2 05M0~ P.S. 046 Art~           2014             8               95
## 3 05M0~ P.S. 046 Art~           2015             8               73
## 4 05M0~ P.S. 046 Art~           2016             8               56
## 5 05M1~ P.S. 123 Mah~           2013             8               62
## 6 05M1~ P.S. 123 Mah~           2014             8               62
## # ... with 2 more variables: `Number of students who registered for the
## #   SHSAT` <int>, `Number of students who took the SHSAT` <int>
\end{verbatim}

\begin{verbatim}
## # A tibble: 6 x 161
##   `Adjusted Grade` `New?` `Other Location~ `School Name` `SED Code`
##   <chr>            <chr>  <chr>            <chr>              <dbl>
## 1 <NA>             <NA>   <NA>             P.S. 015 ROB~    3.10e11
## 2 <NA>             <NA>   <NA>             P.S. 019 ASH~    3.10e11
## 3 <NA>             <NA>   <NA>             P.S. 020 ANN~    3.10e11
## 4 <NA>             <NA>   <NA>             P.S. 034 FRA~    3.10e11
## 5 <NA>             <NA>   <NA>             THE STAR ACA~    3.10e11
## 6 <NA>             <NA>   <NA>             P.S. 064 ROB~    3.10e11
## # ... with 156 more variables: `Location Code` <chr>, District <int>,
## #   Latitude <dbl>, Longitude <dbl>, `Address (Full)` <chr>, City <chr>,
## #   Zip <int>, Grades <chr>, `Grade Low` <chr>, `Grade High` <chr>,
## #   `Community School?` <chr>, `Economic Need Index` <chr>, `School Income
## #   Estimate` <chr>, `Percent ELL` <chr>, `Percent Asian` <chr>, `Percent
## #   Black` <chr>, `Percent Hispanic` <chr>, `Percent Black /
## #   Hispanic` <chr>, `Percent White` <chr>, `Student Attendance
## #   Rate` <chr>, `Percent of Students Chronically Absent` <chr>, `Rigorous
## #   Instruction %` <chr>, `Rigorous Instruction Rating` <chr>,
## #   `Collaborative Teachers %` <chr>, `Collaborative Teachers
## #   Rating` <chr>, `Supportive Environment %` <chr>, `Supportive
## #   Environment Rating` <chr>, `Effective School Leadership %` <chr>,
## #   `Effective School Leadership Rating` <chr>, `Strong Family-Community
## #   Ties %` <chr>, `Strong Family-Community Ties Rating` <chr>, `Trust
## #   %` <chr>, `Trust Rating` <chr>, `Student Achievement Rating` <chr>,
## #   `Average ELA Proficiency` <chr>, `Average Math Proficiency` <chr>,
## #   `Grade 3 ELA - All Students Tested` <int>, `Grade 3 ELA 4s - All
## #   Students` <int>, `Grade 3 ELA 4s - American Indian or Alaska
## #   Native` <int>, `Grade 3 ELA 4s - Black or African American` <int>,
## #   `Grade 3 ELA 4s - Hispanic or Latino` <int>, `Grade 3 ELA 4s - Asian
## #   or Pacific Islander` <int>, `Grade 3 ELA 4s - White` <int>, `Grade 3
## #   ELA 4s - Multiracial` <int>, `Grade 3 ELA 4s - Limited English
## #   Proficient` <int>, `Grade 3 ELA 4s - Economically
## #   Disadvantaged` <int>, `Grade 3 Math - All Students tested` <int>,
## #   `Grade 3 Math 4s - All Students` <int>, `Grade 3 Math 4s - American
## #   Indian or Alaska Native` <int>, `Grade 3 Math 4s - Black or African
## #   American` <int>, `Grade 3 Math 4s - Hispanic or Latino` <int>, `Grade
## #   3 Math 4s - Asian or Pacific Islander` <int>, `Grade 3 Math 4s -
## #   White` <int>, `Grade 3 Math 4s - Multiracial` <int>, `Grade 3 Math 4s
## #   - Limited English Proficient` <int>, `Grade 3 Math 4s - Economically
## #   Disadvantaged` <int>, `Grade 4 ELA - All Students Tested` <int>,
## #   `Grade 4 ELA 4s - All Students` <int>, `Grade 4 ELA 4s - American
## #   Indian or Alaska Native` <int>, `Grade 4 ELA 4s - Black or African
## #   American` <int>, `Grade 4 ELA 4s - Hispanic or Latino` <int>, `Grade 4
## #   ELA 4s - Asian or Pacific Islander` <int>, `Grade 4 ELA 4s -
## #   White` <int>, `Grade 4 ELA 4s - Multiracial` <int>, `Grade 4 ELA 4s -
## #   Limited English Proficient` <int>, `Grade 4 ELA 4s - Economically
## #   Disadvantaged` <int>, `Grade 4 Math - All Students Tested` <int>,
## #   `Grade 4 Math 4s - All Students` <int>, `Grade 4 Math 4s - American
## #   Indian or Alaska Native` <int>, `Grade 4 Math 4s - Black or African
## #   American` <int>, `Grade 4 Math 4s - Hispanic or Latino` <int>, `Grade
## #   4 Math 4s - Asian or Pacific Islander` <int>, `Grade 4 Math 4s -
## #   White` <int>, `Grade 4 Math 4s - Multiracial` <int>, `Grade 4 Math 4s
## #   - Limited English Proficient` <int>, `Grade 4 Math 4s - Economically
## #   Disadvantaged` <int>, `Grade 5 ELA - All Students Tested` <int>,
## #   `Grade 5 ELA 4s - All Students` <int>, `Grade 5 ELA 4s - American
## #   Indian or Alaska Native` <int>, `Grade 5 ELA 4s - Black or African
## #   American` <int>, `Grade 5 ELA 4s - Hispanic or Latino` <int>, `Grade 5
## #   ELA 4s - Asian or Pacific Islander` <int>, `Grade 5 ELA 4s -
## #   White` <int>, `Grade 5 ELA 4s - Multiracial` <int>, `Grade 5 ELA 4s -
## #   Limited English Proficient` <int>, `Grade 5 ELA 4s - Economically
## #   Disadvantaged` <int>, `Grade 5 Math - All Students Tested` <int>,
## #   `Grade 5 Math 4s - All Students` <int>, `Grade 5 Math 4s - American
## #   Indian or Alaska Native` <int>, `Grade 5 Math 4s - Black or African
## #   American` <int>, `Grade 5 Math 4s - Hispanic or Latino` <int>, `Grade
## #   5 Math 4s - Asian or Pacific Islander` <int>, `Grade 5 Math 4s -
## #   White` <int>, `Grade 5 Math 4s - Multiracial` <int>, `Grade 5 Math 4s
## #   - Limited English Proficient` <int>, `Grade 5 Math 4s - Economically
## #   Disadvantaged` <int>, `Grade 6 ELA - All Students Tested` <int>,
## #   `Grade 6 ELA 4s - All Students` <int>, `Grade 6 ELA 4s - American
## #   Indian or Alaska Native` <int>, `Grade 6 ELA 4s - Black or African
## #   American` <int>, ...
\end{verbatim}

Our data looks pretty clean by most standards, but there is work to be
done for sure. For example, we'll need to rename some of our variable
names, join our two tables, etc. Let's move forward with some cleaning.

\subsection{Data Cleaning}\label{data-cleaning}

In order to make our data analysis easier, let's start cleaning the
data. Let's begin by focusing on \texttt{reg}.

\subsubsection{Reg}\label{reg}

\section{Finding the average number of students at each school, how many
registered, and how many took the
test}\label{finding-the-average-number-of-students-at-each-school-how-many-registered-and-how-many-took-the-test}

reg\textless{}- reg \%\textgreater{}\% group\_by(school)
\%\textgreater{}\% mutate\_if(is.numeric, mean) \%\textgreater{}\%
mutate\_if(is.character, funs(paste(unique(.), collapse = ``\_``)))
\%\textgreater{}\% distinct()

reg\(school<-toupper(reg\)school)

\section{Let's remove some columns that have no use for our future
analysis.}\label{lets-remove-some-columns-that-have-no-use-for-our-future-analysis.}

reg\textless{}-reg \%\textgreater{}\% select(-year )\%\textgreater{}\%
select(-grade)

\section{Calculating percentages of those who registered of the total
population, registered AND took the test, and those who took the test of
the total
population}\label{calculating-percentages-of-those-who-registered-of-the-total-population-registered-and-took-the-test-and-those-who-took-the-test-of-the-total-population}

reg\textless{}-reg\%\textgreater{}\% mutate(regPct =
registered/enroll)\%\textgreater{}\% mutate(tookPct =
took/enroll)\%\textgreater{}\% mutate(yield = took/registered)

\section{What does the table currently look
like?}\label{what-does-the-table-currently-look-like}

head(reg)

\section{what does the table look
like?}\label{what-does-the-table-look-like}

head (school)

\section{remove unneccessary columns}\label{remove-unneccessary-columns}

school\textless{}-school\%\textgreater{}\%
select(-\texttt{Adjusted\ Grade})\%\textgreater{}\%
select(-\texttt{New?})\%\textgreater{}\%
select(-\texttt{Other\ Location\ Code\ in\ LCGMS})\%\textgreater{}\%
select(-\texttt{SED\ Code})

\section{renaming some columns}\label{renaming-some-columns}

head(school)

school\textless{}- school \%\textgreater{}\% rename(school =
\texttt{School\ Name}, address = \texttt{Address\ (Full)}, DBN =
\texttt{Location\ Code}, lat = Latitude, long = Longitude, zip = Zip,
eni = \texttt{Economic\ Need\ Index}, district = District, commSchool =
\texttt{Community\ School?}, income = \texttt{School\ Income\ Estimate},
pctELL = \texttt{Percent\ ELL}, pctAsian = \texttt{Percent\ Asian},
pctBlack = \texttt{Percent\ Black}, pctBlackHispanic =
\texttt{Percent\ Black\ /\ Hispanic}, pctHispanic =
\texttt{Percent\ Hispanic}, pctWhite = \texttt{Percent\ White},
pctAttend = \texttt{Student\ Attendance\ Rate}, pctAbsentChronic =
\texttt{Percent\ of\ Students\ Chronically\ Absent}, pctRigor =
\texttt{Rigorous\ Instruction\ \%}, ratingRigor =
\texttt{Rigorous\ Instruction\ Rating}, pctCollab =
\texttt{Collaborative\ Teachers\ \%}, ratingCollab =
\texttt{Collaborative\ Teachers\ Rating}, pctSupp =
\texttt{Supportive\ Environment\ \%}, ratingSupp =
\texttt{Supportive\ Environment\ Rating}, pctLeader =
\texttt{Effective\ School\ Leadership\ \%}, ratingLeader =
\texttt{Effective\ School\ Leadership\ Rating}, pctCommunity =
\texttt{Strong\ Family-Community\ Ties\ \%}, ratingCommunity =
\texttt{Strong\ Family-Community\ Ties\ Rating}, pctTrust =
\texttt{Trust\ \%}, ratingTrust = \texttt{Trust\ Rating}, avgELA =
\texttt{Average\ ELA\ Proficiency}, avgMath =
\texttt{Average\ Math\ Proficiency})

\section{now we want to join the tables
together}\label{now-we-want-to-join-the-tables-together}

school\_clean \textless{}- full\_join(school, reg, by = ``DBN'' )

school\_clean \textless{}- school\_clean\%\textgreater{}\%
filter(!is.na(school.x))\%\textgreater{}\%
select(-school.y)\%\textgreater{}\% rename(school = school.x)

\section{let's turn some qualitative ratings into quantitative scores.
First, let's create a function to make creating these new columns
easier.}\label{lets-turn-some-qualitative-ratings-into-quantitative-scores.-first-lets-create-a-function-to-make-creating-these-new-columns-easier.}

quantScore \textless{}- function(x)\{ if\_else(grepl(``Not Meeting
Target'', x),1 , if\_else(grepl(``Approaching Target'', x),2 ,
if\_else(grepl(``Meeting Target'', x),3 , if\_else(grepl(``Exceeding
Target'', x), 4, 0))))

\}

school\_clean\textless{}- school\_clean\%\textgreater{}\%
mutate(quantRigor = quantScore(ratingRigor)) \%\textgreater{}\%
mutate(quantCollab = quantScore(ratingCollab)) \%\textgreater{}\%
mutate(quantSupp = quantScore(ratingSupp)) \%\textgreater{}\%
mutate(quantLeader = quantScore(ratingLeader)) \%\textgreater{}\%
mutate(quantCommunity = quantScore(ratingCommunity)) \%\textgreater{}\%
mutate(quantTrust = quantScore(ratingTrust))

write\_csv(school\_clean, ``SHSAT\_data.csv'')


\end{document}
